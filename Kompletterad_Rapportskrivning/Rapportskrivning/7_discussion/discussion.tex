% this file is called up by thesis.tex
% content in this file will be fed into the main document

\chapter{Lag och Etik} % top level followed by section, subsection


% ----------------------- paths to graphics ------------------------

% change according to folder and file names
\ifpdf
    \graphicspath{{7/figures/PNG/}{7/figures/PDF/}{7/figures/}}
\else
    \graphicspath{{7/figures/EPS/}{7/figures/}}
\fi


% ----------------------- contents from here ------------------------

I Sverige finns det regler och lagar som gäller för kameraövervakning. Kameraövervakningslagen (2013:460) omfattar dels övervakningskameror, dels tekniska anordningar för att behandla eller bevara bilder och andra tekniska anordningar för avlyssning eller upptagning av ljud som används i samband med övervakningskameror \cite{lansstyrelsen}. 
Enligt lagen är att tillstånd krävs om:

\begin{itemize}
\item kameran riktas mot "en plats dit allmänheten har tillträde"

\item utrustningen kan användas för personbevakning

\item kameran är uppsatt utan att manövereras på platsen
\end{itemize}
I definitionen ”allmänheten har tillträde” tar man ingen hänsyn till om det handlar om privat eller allmän mark, utan alla platser dit allmänheten någon gång har tillträde omfattas av lagen om allmän övervakning. Busshållplatser och gator räknas som allmänna platser enligt definitionen av allmän plats.  Det här ställer vissa krav på den som installerar och/eller äger det system som vi har skapat under detta projektet. Man måste se till att lagar och regler följs dvs att man måste göra en ansökan om tillstånd till allmän kameraövervakning.\\
Kameraövervakning är en metod som används för att minska brottslighet. Effekterna varierar beroende på hur man arbetar med kamerorna. Kameraövervakning är en känslig fråga utifrån ett integritetsperspektiv. Det som människor är oroliga för när det gäller övervakningskameror är att deras integritet kränks. Ingenjörer har ansvar att verka för att tekniken används för samhällets och mänsklighetens bästa enligt Hederskodexen för Sveriges Ingenjörer \cite{sverige}. Vi använder kameraövervakningen i syfte att bekämpa brott vilket är bra för både samhället och människorna i det samhället.




% ---------------------------------------------------------------------------
% ----------------------- end of thesis sub-document ------------------------
% ---------------------------------------------------------------------------

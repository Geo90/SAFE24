
% Thesis Abstract -----------------------------------------------------


%\begin{abstractslong}    %uncommenting this line, gives a different abstract heading

\begin{abstracts}        %this creates the heading for the abstract page
Mot slutet på kursen Inbyggda System och Signaler fick studenter som läser andra året, ett examinationsarbete i samarbete med Axis i Lund. De fick låna en avancerad och en kostsam IP-kamera, samt ett litet elektriskt kit med en mikroprocessor och några sensorer. Projektet inleddes med en föreläsning från två Axis-anställda som sedan klargjorde att examinationsuppgiften var att komma på ett problem och en lösning till det problemet. En grupp, SAFE24, bestod av Louay, Yurdaer, George och Benjamin som sedan tidigare studier under utbildningens gång var bekanta med varandra. Det fanns olika idéer i gruppen om vad problemet skulle vara. Efter några möten kom det fram ett problem i samhället. Lösningen byggde främst på de komponenter som gruppen fick, dock fanns det utrymme för att köpa ytterligare fler förutsatt att det inte översteg en viss budget.



\begin{figure}[h]
  \includegraphics[width=\linewidth]{group1.jpg}
  \caption{Gruppen SAFE24 (www.gaia3d.co.uk/about/ redigerad)}
  \label{fig:group1}
\end{figure}


\end{abstracts}
%\end{abstractlongs}


% ---------------------------------------------------------------------- 

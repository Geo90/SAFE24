
% this file is called up by thesis.tex

\begin{document}


\chapter{Material \& metoder} % top level followed by section, subsection
\label{ch:metoder}

% ----------------------- paths to graphics ------------------------

% change according to folder and file names
\ifpdf
    \graphicspath{{8/figures/PNG/}{8/figures/PDF/}{8/figures/}}
\else
    \graphicspath{{8/figures/EPS/}{8/figures/}}
\fi

% ----------------------- contents from here ------------------------



\section{Metoder}
För att hela systemet skulle kunna fungera samtidigt så användes en schemaläggare så att varje komponent kunde vara aktiv utan att begränsa andra komponenter. Detta var nödvändigt att göra eftersom sensorer som användes skulle lyssna kontinuerligt på förändringar hos omgivningen medan systemet var aktivt och utförde andra uppgifter.\\

En server upprättades för att kunna ta emot data från kameran och lagra denna datan på en dator.\\

All kodning gjordes i Arduino IDE och olika bibliotek användes för att ansluta till ett WiFi-nätverk, skapa en schemaläggare, och kommunicera med kameran via HTTP.\\ 

\section{ESP8266 - mikroprocessorn}
Huvudanledningen för valet av denna mikroprocessor var att den hade en inbyggd Wifi-mottagare vilket var absolut nödvändigt för att kunna kommunicera med nätverket där kameran är upppkopplad. Givetvis kunde vi hitta på alternativa lösningar men just denna lösning var den smidigaste. I övrigt fanns det allt vi behövde till vårt projekt. Det fanns gott om digitala pins samt en analog på max, dock tog den emot max 1 V. \\

Vår tanke var att kopppla sensorer till vår ESP som sedan kommunicerade med kameran utifrån de uppgiffter som lästes in från sensorerna. ESP:n var då kopplad till samma nätverk som kameran och kommunicerade och skickade kommandon till kameran.\\

\section{IP-kameran}
Kameran som används är tillhandahållen av AXIS. Kameran är en IP-kamera med möjlighet till internet uppkoppling sända bilder och streama video till en server.\\

IP-kameran användes för att skicka bilder och video till en server för datalagring.\\

Kommunikation med kameran gjordes via ESP8266 som sände kommandon över internet för att styra kameran.\\

\section{Sensorer}
En PIR-sensor och en mikrofon användes.\\

PIR-sensorn registrerade ifall det förekom någon rörelse. Denna sensorn skickade digitala värden till ESP8266.\\

Mikrofonen som användes skickade analoga värden till ESP8266.\\



% ---------------------------------------------------------------------------
%: ----------------------- end of thesis sub-document ------------------------
% ---------------------------------------------------------------------------



 



\end{document}



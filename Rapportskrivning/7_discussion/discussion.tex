% this file is called up by thesis.tex
% content in this file will be fed into the main document

\chapter{Diskussion \& Framtid} % top level followed by section, subsection
\label{ch:diskussion}

% ----------------------- paths to graphics ------------------------

% change according to folder and file names
\ifpdf
    \graphicspath{{7/figures/PNG/}{7/figures/PDF/}{7/figures/}}
\else
    \graphicspath{{7/figures/EPS/}{7/figures/}}
\fi


% ----------------------- contents from here ------------------------



\section{Diskussion}
I Sverige det finns regler som gäller för kameraövervakning. Kameraövervakningslagen (2013:460) omfattar dels övervakningskameror dels tekniska anordningar för att behandla eller bevara bilder och andra tekniska anordningar för avlyssning eller upptagning av ljud som används i samband med övervakningskameror. 
Enligt huvudregeln är att tillstånd krävs om:

\begin{itemize}
\item[kameran riktas mot "en plats dit allmänheten har tillträde"]

\item[utrustningen kan användas för personbevakning och]

\item[kameran är uppsatt utan att manövreras på platsen]
\end{itemize}
I definitionen ”allmänheten har tillträde” tar man ingen hänsyn till om det handlar om privat eller allmän mark, utan alla platser dit allmänheten någon gång har tillträde omfattats av lagen om allmän övervakning. Busshållplatser och gatorna räknas som allmänt platser enligt definitionen av allmänt plats.  Det här ställer vissa krav på den som installerar och/eller äger det systemet som vi har skapat under den examinations projekt. Man måste se till att lagar och regler följs dvs man måste göra en ansökan om tillstånd till allmän kameraövervakning. 
Kameraövervakning är en metod som används för att minska brottsligheten. Effekterna varierar beroende på hur man arbetar med kamerorna. Andra sidan är kameraövervakning alltid känsligt utifrån ett integritetsperspektiv. Det som är människor oroliga för när det gäller övervakningskameror är att integritet av människors privata liv.  Det är människors privatliv som människor har rätt beskydda. En ingenjör bör respektera detta som alla andra människor. Ingenjörer har ansvar att verka för att tekniken används för samhällets och mänsklighetens bästa enligt hederskodexen för Sveriges ingenjörer. Vi använder kameraövervakningen med ett syfte som är att bekämpa brott. 




% ---------------------------------------------------------------------------
% ----------------------- end of thesis sub-document ------------------------
% ---------------------------------------------------------------------------
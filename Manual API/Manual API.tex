%%%%%%%%%%%%%%%%%%%%%%%%%%%%%%%%%%%%%%%%%
% University/School Laboratory Report
% LaTeX Template
% Version 3.1 (25/3/14)
%
% This template has been downloaded from:
% http://www.LaTeXTemplates.com
%
% Original author:
% Linux and Unix Users Group at Virginia Tech Wiki 
% (https://vtluug.org/wiki/Example_LaTeX_chem_lab_report)
%
% License:
% CC BY-NC-SA 3.0 (http://creativecommons.org/licenses/by-nc-sa/3.0/)
%
%%%%%%%%%%%%%%%%%%%%%%%%%%%%%%%%%%%%%%%%%

%----------------------------------------------------------------------------------------
%	PACKAGES AND DOCUMENT CONFIGURATIONS
%----------------------------------------------------------------------------------------

\documentclass[11pt, numbers=endperiod]{report}

\usepackage[version=3]{mhchem} % Package for chemical equation typesetting
\usepackage{siunitx} % Provides the \SI{}{} and \si{} command for typesetting SI units
\usepackage{graphicx} % Required for the inclusion of images
%\usepackage{natbib} % Required to change bibliography style to APA
\usepackage{amsmath} % Required for some math elements 
\usepackage[colorlinks=false]{hyperref}
\usepackage[utf8]{inputenc}
\usepackage{titleref}
\usepackage{lipsum, etoolbox}
\usepackage{{titlesec}}

\renewcommand{\ref}{Källförteckning}

\makeatletter
\newcommand*{\currentname}{\TR@currentTitle}
\newcommand*{\fullref}[1]{\hyperref[{#1}]{sektion \ref*{#1}. \nameref*{#1}}} % One single link
%\newcommand*{\fullref}[1]{\nameref*{#1}} % No link
\patchcmd{\l@section}
  {\hfil}
  {\leaders\hbox{\normalfont$\m@th\mkern \@dotsep mu\hbox{.}\mkern \@dotsep mu$}\hfill}
  {}{}
  
  
  \newcommand\frontmatter{%
    \cleardoublepage
  %\@mainmatterfalse
  \pagenumbering{roman}}

\newcommand\mainmatter{%
    \cleardoublepage
 % \@mainmattertrue
  \pagenumbering{arabic}}

\makeatother

\renewcommand{\contentsname}{Innehållsförteckning}
\renewcommand{\index}{sektion}

\titlelabel{\thetitle.\quad}


%----------------------------------------------------------------------------------------
%	
%----------------------------------------------------------------------------------------


\setlength\parindent{0pt} % Removes all indentation from paragraphs

\usepackage{tcolorbox}

\renewcommand\thesection{\color{cyan!}\arabic{section}}

\renewcommand\thesection{\color{cyan!}}

\titleformat{\section}
  {\normalfont\sffamily\Large\bfseries\color{cyan}}
  {\thesection}{1em}{}

\titleformat{\section}
  {\normalfont\Large\bfseries}{\thesection}{1em}{}[{\titlerule[0.8pt]}]
  

%%%%%%%%%%%%%%%%%%%%%%%%%
%		MY COMMANDS		%
%%%%%%%%%%%%%%%%%%%%%%%%%
\newcommand{\Vseparation}{\vspace{10mm}}

\newcommand{\setColor}[1]
			{\color{blue!}\textbf{#1}\color{black!}}
\newcommand\MySection[2][\DefaultOpt]{%
  \def\DefaultOpt{#2}%
  \color{cyan!}\section[#1]{#2}\color{black!}%
}

%----------------------------------------------------------------------------------------
%	DOCUMENT INFORMATION
%----------------------------------------------------------------------------------------


\title{SAFE24 API } % Title

\author{}%George \textsc{Albert Florea}} % Author name

\date{22 Februari, 2017 Malmö} % Date for the report

\begin{document}

\begin{titlepage}


\clearpage\maketitle % Insert the title, author and date
\thispagestyle{empty}

\begin{figure}[h!]
	\begin{center}
		\includegraphics[width=0.25\textwidth]{MAH_logotyp_original}
	\end{center}
\end{figure}

\begin{center}
\begin{tabular}{l r}
Last update: & 19 February, 2017 \\ % Date the experiment was performed
\end{tabular}
\end{center}

\end{titlepage}

\clearpage

% Add a link target to the TOC itself
\addtocontents{toc}{\protect\hypertarget{toc}{}}

\tableofcontents

\thispagestyle{empty}\newpage
\setcounter{page}{1}


%----------------------------------------------------------------------------------------
%	Inledning
%----------------------------------------------------------------------------------------
\section{Inledning}
Detta API förklarar och framhäver funktionaliteten hos programkoden. Programkodens möjlighet att kunna implementeras i andra projekt och dess struktur kommer att framgå tydligt. Genom denna manual så kommer användare att lättare kunna förstå och snabbt sätta sig in i vilka klasser och metoder som finns tillgängliga.\\

\hyperlink{toc}{Table of Contents}




%%%%%%%%%%%%%%%%%%%%%%%%%
%     main.ino     		%
%%%%%%%%%%%%%%%%%%%%%%%%%

\MySection{main.ino}
\label{sec: main.ino}
%%%%%%%%%%%%%%%%%%%%%%%%%%%%%%%%%%%%%%%
\begin{tcolorbox}[colback=white,colframe=cyan,width=\dimexpr\textwidth+12mm\relax,enlarge left by=-6mm]

Initializes all constants needed such as SSID and password in order to connect to a wifi-source. Connects to a wifi-source and also initiates the three tasks that are presented below.
\subsection*{class PirTask : public Task}
Scheduled task that uses the PIR-sensor to detect movement.\\
\begin{tcolorbox}[colback=white,colframe=cyan,width=\dimexpr\textwidth+4mm\relax,enlarge left by=-2mm]


\subsubsection*{doWithPirValue(int)}
Uses a reading from pir sensor to calculate a movement\\

\textbf{Return:} \setColor{void}
\Vseparation
\subsubsection*{doWhenMove()}
Checks if there is a movement or not and acts according to that\\

\textbf{Return:} \setColor{void}
\Vseparation

\end{tcolorbox}

\subsection*{class MicTask : public Task}
Scheduled task that listens to the microphone sensor in order to detect sound.\\
\begin{tcolorbox}[colback=white,colframe=cyan,width=\dimexpr\textwidth+4mm\relax,enlarge left by=-2mm]

\subsubsection*{doWithSensorValue(int)}
Checks if the sensor value goes past the threshold of 75.\\

\textbf{Return:} \setColor{void}
\Vseparation


\end{tcolorbox}

\subsection*{class WifiTask : public Task}
Scheduled task that ensures that there is a wifi-connection. This task checks the wifi-connecteion periodically.\\

\end{tcolorbox}

\hyperlink{toc}{Table of Contents}


%%%%%%%%%%%%%%%%%%%%%%%%%
%     connectWifi.h     %
%%%%%%%%%%%%%%%%%%%%%%%%%
\MySection{connectWifi.h}
\label{sec: connectWifi.h}
%%%%%%%%%%%%%%%%%%%%%%%%%%%%%%%%%%%%%%%
\begin{tcolorbox}[colback=white,colframe=cyan,width=\dimexpr\textwidth+12mm\relax,enlarge left by=-6mm]


\subsection*{connectWifi(const char*, const char*)}
Function that establishes a connection to the specified host\\

\textbf{Return:} \setColor{void}
\Vseparation

\subsection*{checkConnection()}
Function that establishes a connection to the specified host\\

\textbf{Return:} \setColor{int}
\Vseparation


\end{tcolorbox}

\hyperlink{toc}{Table of Contents}


%%%%%%%%%%%%%%%%%%%%%%%%%
%    manageCamera.h     %
%%%%%%%%%%%%%%%%%%%%%%%%%
\MySection{manageCamera.h}
\label{sec: manageCamera.h}
%%%%%%%%%%%%%%%%%%%%%%%%%%%%%%%%%%%%%%%
\begin{tcolorbox}[colback=white,colframe=cyan,width=\dimexpr\textwidth+12mm\relax,enlarge left by=-6mm]


\subsection*{sendToCamera ( String, String, const char*, const char*)}
Sending commands to the camera\\

\textbf{Return:} \setColor{int}
\Vseparation


\subsection*{continuousPanTiltMove (int, int, int)}
Moves the cames depending on the arguments.\\
Positive values mean right (pan) and up (tilt),\\
  negative values mean left (pan) and down (tilt),\\
	"0,0" means stop
\textbf{Return:} \setColor{String}
\Vseparation

\subsection*{activateVirtualPort (String)}
Activates a virtual port on Camera\\

\textbf{Return:} \setColor{String}
\Vseparation

\subsection*{activateVirtualPort (String)}
Dectivates a virtual port on Camera\\

\textbf{Return:} \setColor{String}
\Vseparation
\end{tcolorbox}



\hyperlink{toc}{Table of Contents}
\end{document}